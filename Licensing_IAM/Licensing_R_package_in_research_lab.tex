\documentclass[12pt, colorinlistoftodos]{article} 

\title{Licensing R packages developed in a research unit}
\author{Maxime Jaunatre}

\usepackage{ifthen} 
\newboolean{draft}\setboolean{draft}{false} % display notes, stamp

\providecommand{\main}{.}  % *Modification: define file location
\usepackage{\main/.tex/setup}

% \usepackage{draftwatermark}
%     \SetWatermarkScale{4}
%     \SetWatermarkText{DRAFT}
%     \SetWatermarkLightness{0.8}

\begin{document}

\thispagestyle{plain}

\begin{figure}
    \includegraphics[width=\textwidth]{\main/.tex/header_ifre.png}
    \par ~ \par
    \begin{minipage}{\textwidth}
        \begin{center}
            {\huge \csname @title\endcsname }
        \end{center}
        \rule{7em}{.4pt}\par
        \csname @author\endcsname ~| UMR AMURE \hfill %\par 
        \href{mailto:maxime.jaunatre@ifremer.fr}{Mail} | \today
    \end{minipage}
\end{figure}
\hrule
%\ifthenelse{\boolean{draft}}{\listoftodos \hrule}{} %is this list present or not

\section{Introduction}

Every line of code produced as part of the research is subject to copyright. 
However, copyright only allows the author to make decisions about modifying, sharing and using his software.
A license is therefore essential to declare precisely the conditions of use/communication/modification of the software by others.

This document gathers some elements of knowledge and thoughts used to determine the choice of licensing the R packages produced in the framework of the IAM bio-economic model development within the UMR AMURE. 
It includes considerations related to the licensing of software produced in the framework of research work.
%It also reproduces the process undertaken in the framework of IAM to select its license.

\section{Licenses}

Licenses are contracts protecting authors, collaborators and eventually users. 
They can establish obligations to respect for the sharing/use/modification of the software.

It is therefore essential to license developed software as soon as possible and before distribution.
Licenses establish obligations. The use of other software (packages, dependencies, bricks)
can therefore lead to having to respect other licenses.
It is therefore necessary to pay attention to the compatibility of the licenses.
\textit{Example: if my software uses a package whose license prevents a commercial use, I could not market my software.}

An essential concept is copyleft (as opposed to copyright). 
The copyleft of a license allows to determine the amount of permissions of a software.
Thus a strong copyleft imposes reciprocity and maintenance of the original license with modifications, 
while a weak copyleft allows modifications to be under other licenses.
In the case of a strong copyleft, we say that the initial license contaminates the license of my software.

\subsection*{Free software}

A software is said to be free if it validates 4 criteria according to 
the \href{https://www.gnu.org/philosophy/free-sw.html#four-freedoms}{Free Software Foundation (FSF)}:

\begin{enumerate}
    \item freedom to run the program
    \item freedom to study how the program works and to modify it (access to the source code)
    \item freedom to distribute copies to others
    \item freedom to distribute modified copies (with source code)
\end{enumerate}

It is more a philosophy of sharing than a notion of price.

\subsection*{Open-source software}

An software is open source if its license meets these conditions according 
to the \href{https://opensource.org/docs/osd}{Open Source Initiative (OSI)}:

\begin{enumerate}
    \item Free redistribution. The license must not prevent a party from selling or giving away the software
    \item Source code (availability of the code)
    \item Derived works
    \item Integrity of the author's source code
    \item No discrimination against individuals or groups
    \item No discrimination against fields of activity
    \item Distribution of the license
    \item The license must not be product specific
    \item The license must not restrict other software
    \item The license must be technology neutral
\end{enumerate}

\section{Licensing R}

As mentioned earlier, some licenses may be incompatible.
This is important in the case of an R package, since the R language is under a license itself.

Therefore, it is necessary to understand which licenses are used in the context of R.
A sum up of information by Colin Fay (ThinkR dev) is accessible in \href{https://thinkr-open.github.io/licensing-r/}{\textit{Licensing R} (Fay Colin, 2019).}

To sum up, R require the license to be in a list they provide and GNU GPL\footnote{acronym for GNU General Public License, GNU being a project bringing together several free software (acronym : \textit{"GNU’s Not UNIX"}).}-compatible.
This is because R packages are considered as add-on or plug-in, and simplify the list of licenses we can use.

\textit{Licensing R} contains an important part about checking dependancies'licenses. 
The R code is provided in the \href{https://thinkr-open.github.io/licensing-r/practical.html#dependencies-exploration}{5.2 section}.

\section{Ifremer context}

Institutions where the code is developped has its importance.
Development produced in public institutions as Ifremer have to comply to 
\href{https://www.legifrance.gouv.fr/loda/article_lc/LEGIARTI000033205142/2020-09-21/}{"La loi numérique"}. 
This mean that every software produced with public funding must be 
under open-source license, free and accessible to everyone.
In order to frame this, the french governement published a \href{https://www.data.gouv.fr/fr/pages/legal/licences/}{list of licenses to use}.
% Donc déjà, sache que vous pouvez déposer le code à l'APP (https://www.app.asso.fr/), ça ne mange pas de pain...

An important point here is the presence of a private funding. 
If so, a restricted license can be used. 
To use a open-source license, you need to collect agrement from the funding organism.

\subsection*{CeCILL, a french license}

In 2004, french agencies (CEA, CNRS and INRIA) announced a new free-software license designed for compatibility between GNU GPL and French law.
The \href{https://cecill.info/index.fr.html}{CeCILL license}\footnote{acronym for Ce(A)C(NRS)I(NRIA)L(ogiciel)L(ibre)}, exist in different versions.

\begin{enumerate}
    \item CeCILL : free software license compatible with GNU GPL, in version 1, 2 and 2.1
    \item CeCILL-B : similar to BSD\footnote{acronym for Berkeley Software Distribution} type license and thus compatible with GNU LGPL\footnote{acronym for GNU Lesser General Public License} licenses (less restrictive licenses)
    \item CeCILL-C : adapted to software component licenses (more restrictive licenses)
\end{enumerate}

\section{Important notes}

On a project involving multiple authors, it is important to have each 
contributor persmission to change the license.
This is because every code line added is under contributor's copyright. 
It is however possible for contributors to assign their copyright to the main main author. 
This will prevent future issues in a growing number of contribution.

The exploration of the different types of licences has thus led to the choice of a CeCILL v2.1 licence for the IAM bio-economic model, 
in order to respect the context of R software and the "Loi numérique".
IAM is thus a free open-source software and the acceptance of the licence is done at the download or at the first application of one of the following rights:
\begin{enumerate}
    \item Right to use the software (reproduce, load, display, execute, store and/or study the operation)
    \item Right to make contributions (translate, adapt, arrange, modify and/or reproduce)
    \item Right to distribute (disseminate, transmit, communicate on any medium and/or by any means as well as to market)
\end{enumerate}
The distribution of IAM with or without modification must always be done with the associated licence (CeCILL) and access to the source code.
IAM can be included and distributed in code licensed under the GNU GPL, GNU Affero GPL or EUPL. 

Translated with www.DeepL.com/Translator (free version)

\section{Links and sources}

\begin{enumerate}
    \item FSF : \url{https://www.gnu.org/philosophy/free-sw.html#four-freedoms}
    \item OSI : \url{https://opensource.org/docs/osd}
    \item \textit{Licensing R}, Fay Colin, 2019 : \url{https://thinkr-open.github.io/licensing-r}
    \item Loi Numérique, Légifrance : \url{https://www.legifrance.gouv.fr/loda/article_lc/LEGIARTI000033205142/2020-09-21/}
    \item data.gouv.fr License list : \url{https://www.data.gouv.fr/fr/pages/legal/licences/}
    \item CeCILL license : \url{https://cecill.info/index.fr.html}
\end{enumerate}

\end{document}