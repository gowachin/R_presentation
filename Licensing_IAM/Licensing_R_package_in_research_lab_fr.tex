\documentclass[12pt, colorinlistoftodos]{article} 

\title{Mettre une licence sur les packages R développés dans une unité de recherche}
\author{Maxime Jaunatre}

\usepackage{ifthen} 
\newboolean{draft}\setboolean{draft}{false} % display notes, stamp

\providecommand{\main}{.}  % *Modification: define file location
\usepackage{\main/.tex/setup}

% \usepackage{draftwatermark}
%     \SetWatermarkScale{4}
%     \SetWatermarkText{DRAFT}
%     \SetWatermarkLightness{0.8}

\begin{document}

\thispagestyle{plain}

\begin{figure}
    \includegraphics[width=\textwidth]{\main/.tex/header_ifre.png}
    \par ~ \par
    \begin{minipage}{\textwidth}
        \begin{center}
            {\huge \csname @title\endcsname }
        \end{center}
        \rule{7em}{.4pt}\par
        \csname @author\endcsname ~| UMR AMURE \hfill %\par 
        \href{mailto:maxime.jaunatre@ifremer.fr}{Mail} | \today
    \end{minipage}
\end{figure}
\hrule
%\ifthenelse{\boolean{draft}}{\listoftodos \hrule}{} %is this list present or not

\section{Introduction}

Chaque ligne de code produit dans le cadre de la recherche est soumis aux règles régissant les droits d'auteurs. 
Cependant, le droit d'auteur permet uniquement à l'auteur de prendre les décisions concernant la modification, le partage et l'utilisation de son logiciel.
Une licence est donc indispensable pour déclarer précisément les conditions d'utilisation/ communication/modification du logiciel par d'autres personnes.

Ce document rassemble quelques éléments de connaissances et réflexions utilisés pour déterminer le choix de la licence 
à déposer sur les packages R produits dans le cadre du développement du modèle bio-économique IAM au sein de l'UMR AMURE. 
Il inclut des considérations relatives à la mise sous licences pour les logiciels produits dans le cadre de travaux de recherche.
%Il reproduit également le processus entrepris dans le cadre d'IAM pour choisir sa licence.

\section{Licences}

Les licences sont des contrats protégeant les auteurs, collaborateurs et éventuellement les utilisateurs.
Elles peuvent établir des obligations à respecter pour le partage/l'utilisation/la modification des logiciels.

Il est donc essentiel de mettre sous licence les logiciels développés dès que possible et ceci avant leur distribution.
Les licences établissent des obligations. L'utilisation d'autres logiciels (packages, dépendances, briques)
peut donc amener à devoir respecter d'autres licences.
Il est alors nécessaire de faire attention à la compatibilité des licences.
\textit{Exemple : si mon logiciel utilise un package dont la licence empêche son utilisation commerciale, je ne peux commercialiser mon logiciel.}

Un concept essentiel à connaitre est le copyleft (opposé au copyright, traduit copie laissée). 
Il s'agit d'un concept associé à certaines licences où l'auteur empêche un ajout futur de restrictions sur 
la distribution, l'utilisation et la modification de son oeuvre. 
Ainsi, un copyleft fort impose la réciprocité et le maintien de la licence originelle avec les modifications,
tandis qu'un copyleft faible autorise les modifications à se faire sous d'autres licences (potentiellement moins libres).
Dans le cas d'un copyleft fort, on dit que la licence initiale contamine la licence de mon logiciel.

\subsection*{Logiciel libre}

Un logiciel est dit libre (free) si il valide 4 critères selon la 
\href{https://www.gnu.org/philosophy/free-sw.html#four-freedoms}{Free Software Foundation (FSF)}:

\begin{enumerate}
    \item liberté d'exécuter le programme
    \item liberté d'étudier le fonctionnement du programme et de le modifier (accès au code source)
    \item liberté de distribuer des copies
    \item liberté de distribuer des copies modifiées (avec le code source)
\end{enumerate}

Il s'agit donc plus d'une philosophie que d'une notion de prix (en anglais free se traduit libre comme gratuit).

\subsection*{Logiciel open-source}

Un logiciel est dit open-source si sa licence valide ces conditions selon la 
\href{https://opensource.org/docs/osd}{Open Source Initiative (OSI)}:

\begin{enumerate}
    \item Redistribution gratuite. La licence ne doit pas empêcher quiconque de vendre ou de donner le logiciel
    \item Code source (disponibilité du code)
    \item Œuvres dérivées
    \item Intégrité du code source de l'auteur
    \item Pas de discrimination à l'égard de personnes ou de groupes
    \item Pas de discrimination à l'égard des domaines d'activité
    \item Distribution de la licence
    \item La licence ne doit pas être spécifique à un produit
    \item La licence ne doit pas restreindre d'autres logiciels
    \item La licence doit être neutre du point de vue technologique
\end{enumerate}

\section{Licences sous R}

Comme indiqué précédemment, certaines licences peuvent être incompatibles.
Ceci est important dans le cadre d'un package R, puisque le language R est lui-même sous licence.

Ainsi, il faut comprendre quelles licences sont utilisées dans le cadre de R.
Un résumé de ces informations par Colin Fay (dev chez ThinkR) est accessible dans \href{https://thinkr-open.github.io/licensing-r/}{\textit{Licensing R} (Fay Colin, 2019).}

Pour résumer, R impose une licence sur une liste disponible et compatible avec la GNU GPL\footnote{acronyme de GNU General Public License, GNU étant un projet rassemblant plusieurs logiciel libres (acronyme : \textit{"GNU’s Not UNIX"}).}. 
Ceci vient du fait que les packages R sont considérés commes des ajouts ou des plug-in, et cela simplifie la liste des licences utilisables.

\textit{Licensing R} contient une partie importante à propos de la vérification des licences utilisées par les dépendances.
Le code R est fournis dans la \href{https://thinkr-open.github.io/licensing-r/practical.html#dependencies-exploration}{section 5.2}.

\section{Contexte de l'Ifremer}

Il est important de faire attention au cadre institutionnel de développement.
Le code développé par les institutions publiques comme l'Ifremer doivent répondre de 
\href{https://www.legifrance.gouv.fr/loda/article_lc/LEGIARTI000033205142/2020-09-21/}{"La loi numérique"}.
Cela signifie que tout logiciel produit avec des fonds publics doit être placé sous licence libre, 
gratuit et accessible par tout le monde.
Pour cadrer cela, le gouvernement français a publié
\href{https://www.data.gouv.fr/fr/pages/legal/licences/}{une liste des licences à utiliser}.
% Donc déjà, sache que vous pouvez déposer le code à l'APP (https://www.app.asso.fr/), ça ne mange pas de pain...

Il faut souligner ici le cas d'un financement en partie privée.
Si c'est le cas, une licence restreinte peut être utilisée.
Pour utiliser une licence open-source, il faut obtenir l'autorisation de l'organisme financeur.

\subsection*{CeCILL, une licence françaises}

En 2004, plusieurs agences françaises (CEA, CNRS et INRIA) ont annoncé une nouvelle licence libre, 
pensée pour être compatible avec la licence GNU GPL et le droit français.
La licence \href{https://cecill.info/index.fr.html}{CeCILL}\footnote{acronyme de Ce(A)C(NRS)I(NRIA)L(ogiciel)L(ibre)}, existe en différentes versions.

\begin{enumerate}
    \item CeCILL : licence libre compatible avec la GNU GPL, en versions 1, 2 et 2.1
    \item CeCILL-B : similaire aux licences type BSD\footnote{acronyme de Berkeley Software Distribution} et compatible avec les licences GNU LGPL\footnote{acronyme de GNU Lesser General Public License} (licences moins restrictives)
    \item CeCILL-C : adapté des licences pour composant logiciel (licences plus restrictives)
\end{enumerate}

\section{Notes importantes}

Pour un projet produit par plusieurs auteurs, il est important
d'avoir la permission de chaque auteur pour changer de licence.
Cela est du au fait que chaque ligne de code est sous le copyright de son auteur.
Il est toutefois possible pour les contributeurs de céder leurs copyright à l'auteur principal.
Cela permet d'éviter de futurs problèmes dans le cas d'un nombre croissant de contributions.

\newpage

\section{Conclusion}

L'exploration des différents types de licences a ainsi conduit au choix d'une licence CeCILL v2.1 pour le modèle bio-économique IAM, 
afin de respecter le contexte du logiciel R ainsi que la "Loi numérique".
IAM est donc un logiciel libre open-source et l'acceptation de la licence se fait au téléchargement ou à la première application d'un des droits suivants :
\begin{enumerate}
    \item Droit d'utilisation du logiciel (reproduire, charger, afficher, exécuter, stocker et/ou étudier le fonctionnement)
    \item Droit d'apporter des contributions (traduire, adapter, arranger, apporter une modification et/ou reproduire)
    \item Droit de distribution (diffuser, transmettre, communiquer sur tout support et/ou par tout moyen ainsi que mettre sur le marché)
\end{enumerate}
La distribution d'IAM avec ou sans modification doit toujours se faire avec la licence associée (CeCILL) et un accès au code source.
IAM peut être inclus et distribué dans un code sous licence GNU GPL, GNU Affero GPL ou EUPL. 

\section{Sources et liens}

\begin{enumerate}
    \item FSF : \url{https://www.gnu.org/philosophy/free-sw.html#four-freedoms}
    \item OSI : \url{https://opensource.org/docs/osd}
    \item \textit{Licensing R}, Fay Colin, 2019 : \url{https://thinkr-open.github.io/licensing-r}
    \item Loi Numérique, Légifrance : \url{https://www.legifrance.gouv.fr/loda/article_lc/LEGIARTI000033205142/2020-09-21/}
    \item Liste de licences data.gouv.fr : \url{https://www.data.gouv.fr/fr/pages/legal/licences/}
    \item Licence CeCILL : \url{https://cecill.info/index.fr.html}
\end{enumerate}


\end{document}
